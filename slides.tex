\documentclass[xcolor=svgnames]{beamer}

\usetheme{Boadilla}
\useoutertheme[subsection=false]{smoothbars}
\usefonttheme{serif}
\usecolortheme{seagull} % dove fly seagull
\useinnertheme{rounded}
\usepackage{pifont}
\usepackage{time}       % date and time
\usepackage{url}
\usepackage{graphicx}
\usepackage[T1]{fontenc}    % european characters
% \usepackage{pgothic}
\usepackage{yfonts}
% \usepackage{courier}
\usepackage{amssymb,amsmath}  % use mathematical symbols
\usepackage{palatino}      % use palatino as the default font
\usepackage{listings}   % insert python code into presentation
\usepackage{multirow}
\setbeamercovered{transparent}
\setbeamertemplate{itemize subitem}[triangle]
\setbeamertemplate{navigation symbols}{}

\newcommand{\tc}{\textcolor}
\newcommand{\select}{$\leadsto~$}
\newcommand{\fb}{\framebox}
\def\ra{$\rightarrow$\,\,}
\definecolor{cadet}{rgb}{0.33, 0.41, 0.47}
\newcommand{\hi}[1]{{\color{cadet}\textsl{#1}}}
\newcommand{\chacha}[1]{\centering{\scalebox{2}{\fontsize{12pt}{0pt}\normalfont{#1}}}}
\newcommand{\chapter}[1]{\begin{frame}{}\chacha{#1}\end{frame}}
\newcommand{\chapterDouble}[2]{\begin{frame}{}\chacha{#1}\\\medskip\chacha{#2}\end{frame}}
\newcommand{\resetEnv}{
  \colorlet{structure}{mystruct}
  \beamertemplateshadingbackground{blue!5}{gray!10}
  \beamertemplateshadingbackground{white}{white}
}
\usepackage{mathtools}
\usepackage{xcolor}
\usepackage{verbatim}
\usepackage{listings}
\usepackage{keystroke}

\definecolor{byzantium}{rgb}{0.44, 0.16, 0.39}
\definecolor{ferrariRed}{rgb}{1.0, 0.11, 0.0}

% tiny-scriptsize-footnotesize-small-normalsize-large-Large-LARGE-huge-Huge

\begin{document}

\title[VisIt Workshop]{\LARGE Parallel programming in Chapel}
\author[]{{\large\sc Juan Zuniga}\\{\small juan.zuniga@usask.ca}\\\bigskip {\large\sc Alex
    Razoumov}\\{\small alex.razoumov@westgrid.ca}}
\date[May 2017]{\textcolor{byzantium}{\footnotesize slides and code examples at at ...}}
\institute[]{\includegraphics[width=0.25\columnwidth]{./logos/ccLogo}
  \hspace{.65cm}\includegraphics[width=0.35\columnwidth]{./logos/WG-Logo-Jan16}}

\begin{frame}
  \titlepage
\end{frame}

\section{Intro}
\subsection{} % creates page markers at the top
\newcommand{\tnij}{\ensuremath{T^{(n)}_{i,j}}}

\begin{frame}{}
  \begin{itemize}\setlength{\itemsep}{3mm}
    \item Simple 2D heat (diffusion) equation
    \[
      {\partial T(x,y,t)\over\partial t}={\partial^2 T\over\partial x^2}+{\partial^2 T\over\partial y^2}
    \]
    {\footnotesize
      \item Discretize the solution $T(x,y,t)\approx T^{(n)}_{i,j}$ with $i=1,...,n$ and $j=1,...,n$
      \item Imagine a metallic plate being constantly heated in one corner, e.g. $T^{(n)}_{1,1}=1$
      \item Everywhere else the initial solution is $T^{(0)}_{i,j}=0$}
    \item Discretize the equation with forward Euler time stepping
    \[
      {T^{(n+1)}_{i,j}-\tnij\over\Delta t}=
      {T^{(n)}_{i+1,j}-2T^{(n)}_{i,j}+T^{(n)}_{i-1,j}\over(\Delta x)^2}+
      {T^{(n)}_{i,j+1}-2T^{(n)}_{i,j}+T^{(n)}_{i,j-1}\over(\Delta y)^2}
    \]
  \end{itemize}
\end{frame}

\begin{frame}{}
  \begin{itemize}\setlength{\itemsep}{3mm}
    \item For simplicity assume $\Delta x=\Delta y=1$
    \item Use $\Delta t=1/4$ which is the upper limit of numerical stability
    \item The finite difference equation becomes
    \[
      T^{(n+1)}_{i,j} = {1\over4}\left[T^{(n)}_{i+1,j}+T^{(n)}_{i-1,j}+T^{(n)}_{i,j+1}+T^{(n)}_{i,j-1}\right]
    \]
    \item The objective is to find $T_{i,j}$ after a
    certain number of iterations, or when the system is in steady state)
    \item Once done, also try increasing the number of points in the grid to illustrate the advantage of
    parallelism
  \end{itemize}
\end{frame}


\section{Chapel base language} % Juan's section
\chapter{Chapel base language}
\subsection{} % creates page markers at the top

\section{Task parallelism} % Juan's section
\chapter{Task parallelism}
\subsection{} % creates page markers at the top

\section{Data parallelism}
\chapter{Data parallelism}
\subsection{} % creates page markers at the top
% revisiting the notion of locality, where in the hardware the tasks (or the different locales) are
% running, parallel data structures, domain maps

\section{Data parallelism}
\chapter{Data parallelism}
\subsection{} % creates page markers at the top

\begin{frame}{}
  \begin{itemize}\setlength{\itemsep}{3mm}
    \item one
    \item two
  \end{itemize}
\end{frame}


\section{Advanced language features}
\chapter{Advanced language features}
\subsection{} % creates page markers at the top
% iterators, I/O files, modules and other packages?
% how to call C/C++/Fortran functions from Chapel, especially popular numerical libraries

\section{Advanced language features}
\chapter{Advanced language features}
\subsection{} % creates page markers at the top

\begin{frame}{}
  \begin{itemize}\setlength{\itemsep}{3mm}
    \item one
    \item two
  \end{itemize}
\end{frame}


\section{Summary}
\subsection{} % creates page markers at the top
\section{Summary}
\subsection{} % creates page markers at the top

\begin{frame}{}
  \begin{itemize}\setlength{\itemsep}{3mm}
    \item one
    \item two
  \end{itemize}
\end{frame}






\end{document}

(1) Juan will teach Chapel base language + task parallelism (language syntax and the compiler, the
notions of locality and parallelism and how they are orthogonal concepts in Chapel, forall and coforall
in abstract level)

(2) Alex will teach Chapel data parallelism + advanced language features (revisiting the notion of
locality, where in the hardware the tasks (or the different locales) are running, parallel data
structures, domain maps, iterators, I/O files, modules and other packages?)
